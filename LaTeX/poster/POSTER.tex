%Eddy
\documentclass[a0paper,portrait]{baposter}

\usepackage[font=small,labelfont=bf]{caption} % Required for specifying captions to tables and figures
\usepackage{booktabs} % Horizontal rules in tables
\usepackage{relsize} % Used for making text smaller in some places

\graphicspath{{figures/}} % Directory in which figures are stored

\definecolor{bordercol}{RGB}{40,40,40} % Border color of content boxes
\definecolor{headercol1}{RGB}{186,215,230} % Background color for the header in the content boxes (left side)
\definecolor{headercol2}{RGB}{80,80,80} % Background color for the header in the content boxes (right side)
\definecolor{headerfontcol}{RGB}{0,0,0} % Text color for the header text in the content boxes
\definecolor{boxcolor}{RGB}{186,215,230} % Background color for the content in the content boxes

\begin{document}

\background{ % Set the background to an image (background.pdf)
\begin{tikzpicture}[remember picture,overlay]
\draw (current page.north west)+(-2em,2em) node[anchor=north west]
{\includegraphics[height=1.1\textheight]{background}};
\end{tikzpicture}
}

\begin{poster}{
grid=false,
borderColor=bordercol, % Border color of content boxes
headerColorOne=headercol1, % Background color for the header in the content boxes (left side)
headerColorTwo=headercol2, % Background color for the header in the content boxes (right side)
headerFontColor=headerfontcol, % Text color for the header text in the content boxes
boxColorOne=boxcolor, % Background color for the content in the content boxes
headershape=roundedright, % Specify the rounded corner in the content box headers
headerfont=\Large\sf\bf, % Font modifiers for the text in the content box headers
textborder=rectangle,
background=user,
headerborder=open, % Change to closed for a line under the content box headers
boxshade=plain
}
{}
%----------------------------------------------------------------------------------------
%	TITLE AND AUTHOR NAME
%----------------------------------------------------------------------------------------
{\sf\bf Some Funny Yet \\ Relevant Title} % Poster title
{\vspace{1em} chriswangzanxu, 14102019, AjmastR, Anon, Or4nge, walnutt\\} % Author names
{\includegraphics[scale=1.1]{the_logo}} % University/lab logo

%----------------------------------------------------------------------------------------
%	INTRODUCTION
%----------------------------------------------------------------------------------------

\headerbox{Introduction}{name=introduction,column=0,row=0}{
Introduction here
}

%----------------------------------------------------------------------------------------
%	AFH
%----------------------------------------------------------------------------------------
\headerbox{Adaptive Frequency Hopping}{name=AFH,column=0,below=introduction}{
Box for AFH (both Bluetooth and Wi-Fi are based on this)
}

%----------------------------------------------------------------------------------------
%	Another misc box
%----------------------------------------------------------------------------------------
\headerbox{Another General Box}{name=AGB,column=0,below=AFH}{
Box for another common feature (add more boxes as needed)
}

%----------------------------------------------------------------------------------------
%	CONCLUSION
%----------------------------------------------------------------------------------------

\headerbox{Conclusion}{name=conclusion,column=0,below=AGB}{
Conclusion here
}

%----------------------------------------------------------------------------------------
%	REFERENCES
%----------------------------------------------------------------------------------------

\headerbox{References}{name=references,column=0,below=conclusion}{

\smaller % Reduce the font size in this block
idk if we need this
\renewcommand{\section}[2]{\vskip 0.05em} % Get rid of the default "References" section title
\nocite{*} % Insert publications even if they are not cited in the poster

\bibliographystyle{unsrt}
\bibliography{sample} % Use sample.bib as the bibliography file
}

%----------------------------------------------------------------------------------------
%	ACKNOWLEDGEMENTS
%----------------------------------------------------------------------------------------

\headerbox{Acknowledgements}{name=acknowledgements,column=0,below=references, above=bottom}{
\smaller % Reduce the font size in this block
Rito
} 

%----------------------------------------------------------------------------------------
%	Wi-Fi
%----------------------------------------------------------------------------------------
\headerbox{Wi-Fi}{name=wifi,span=2,column=1,row=0}{ % To reduce this block to 1 column width, remove 'span=2'
Wi-Fi box here
}

%----------------------------------------------------------------------------------------
%	Optical Fibres
%----------------------------------------------------------------------------------------
\headerbox{Optical Fibres}{name=optical_fibre,span=2,column=1,below=wifi}{
Optical fibre box here
}

%----------------------------------------------------------------------------------------
%	Bluetooth
%----------------------------------------------------------------------------------------
\headerbox{Bluetooth}{name=bluetooth,span=2,column=1,below=optical_fibre}{
Bluetooth box here
}

%----------------------------------------------------------------------------------------
%	Li-Fi
%----------------------------------------------------------------------------------------
\headerbox{Li-Fi}{name=lifi,span=2,column=1,below=bluetooth}{
Li-Fi box here
}

%----------------------------------------------------------------------------------------
%	Neutrinos
%----------------------------------------------------------------------------------------
\headerbox{Neutrino}{name=neutrino,span=2,column=1,below=lifi,above=bottom}{
\textbf{Introduction} \\
Neutrino messaging is a hypothetical form of communication currently undergoing research.
Neutrinos are advantageous in communicative methods in that they pass through normal
matter, but this causes them to be notoriously difficult to detect.\\
\textbf{History}\\
Neutrino messaging was first experimentally verified to work in 2012 by researchers from the
University of Rochester and North Carolina State University.

The message was encoded with on-off keying, with 1 and 0 being represented by the presence
and absence of a neutrino beam pulse, respectively.\\
\textbf{Advantages}\\
Unlike traditional forms of communication which rely on electromagnetic radiation, neutrinos are
affected only by the weak force and gravity, meaning they can pass messages through virtually
anything.

This is ideal for long distance communication - a potential use in the future is sending messages
across vast expanses in space.

A present-day application is sending messages to nuclear submarines. Seawater can obstruct
electromagnetic radiation, so submarines must extend an antenna to the surface, causing them
to be easier to detect.\\
\textbf{Disadvantages}\\
The interactive nature of neutrinos causes them to be difficult to detect. Neutrinos also oscillate
between 3 flavours - electron, muon, and tau.

Neutrinos have both particle and wave properties, so this can be represented by a neutrino
switching between waves of different frequencies as it travels through space. This can be a
problem for certain detection methods.

For example, in the 1960s, there was a discrepancy between the predicted and observed
amount of neutrinos emitted by the sun. This was due to the detector not factoring in that
electron neutrinos from the sun oscillated into different flavours en route to Earth
}

\end{poster}

\end{document}