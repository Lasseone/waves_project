%Eddy
\documentclass[a0paper,portrait]{baposter}
\usepackage[font=small,labelfont=bf]{caption} % Required for specifying captions to tables and figures
\usepackage{booktabs} % Horizontal rules in tables
\usepackage{relsize} % Used for making text smaller in some places
\usepackage{wrapfig} %Used to wrap images around text

\graphicspath{{figures/}} % Directory in which figures are stored

\definecolor{bordercol}{RGB}{40,40,40} % Border color of content boxes
\definecolor{headercol1}{RGB}{186,215,230} % Background color for the header in the content boxes (left side)
\definecolor{headercol2}{RGB}{80,80,80} % Background color for the header in the content boxes (right side)
\definecolor{headerfontcol}{RGB}{0,0,0} % Text color for the header text in the content boxes
\definecolor{boxcolor}{RGB}{186,215,230} % Background color for the content in the content boxes

\begin{document}

\background{ % Set the background to an image (background.pdf)
\begin{tikzpicture}[remember picture,overlay]
\draw (current page.north west)+(-2em,2em) node[anchor=north west]
{\includegraphics[height=1.1\textheight]{background}};
\end{tikzpicture}
}

\begin{poster}{
grid=false,
borderColor=bordercol, % Border color of content boxes
headerColorOne=headercol1, % Background color for the header in the content boxes (left side)
headerColorTwo=headercol2, % Background color for the header in the content boxes (right side)
headerFontColor=headerfontcol, % Text color for the header text in the content boxes
boxColorOne=boxcolor, % Background color for the content in the content boxes
headershape=roundedright, % Specify the rounded corner in the content box headers
headerfont=\Large\sf\bf, % Font modifiers for the text in the content box headers
textborder=rectangle,
background=user,
headerborder=open, % Change to closed for a line under the content box headers
boxshade=plain
}
{}
%----------------------------------------------------------------------------------------
%	TITLE AND AUTHOR NAME
%----------------------------------------------------------------------------------------
{\sf\bf Information \\ Communications} % Poster title
{\vspace{1em} chriswangzanxu, 14102019, AjmastR, Anon, Or4nge, walnutt\\} % Author names
{\includegraphics[scale=1.1]{the_logo}} % University/lab logo

%----------------------------------------------------------------------------------------
%	INTRODUCTION
%----------------------------------------------------------------------------------------

\headerbox{Introduction}{name=introduction,column=0,row=0}{
This poster outlines a variety of technologies, which use waves as a means to transfer information. These are both waves in the conventional sense of electromagnetic radiation, as well as the particle/wave duality sense of neutrinos.
}

%----------------------------------------------------------------------------------------
%	AFH
%----------------------------------------------------------------------------------------
\headerbox{Adaptive Frequency Hopping}{name=AFH,column=0,below=introduction}{
\textbf{What Is Frequency Hopping?} \\
Adaptive Frequency Hopping (AFH) is a tehcnique where rather than using one single radiofrequency to transfer data, the frequency is constantly changing between a number of channels. This allows for both faster transfer speeds, and makes it harder for intruders to interfere with the signal. \\

\textbf{Why Is It Adaptive?} \\
The transmitting device is constantly monitoring the different channels to make an estimate of how good quality they are. For example, if one frequency is currently busy or being jammed, then it will simply use another channel.
}

%----------------------------------------------------------------------------------------
%	Another misc box
%----------------------------------------------------------------------------------------
\headerbox{Another General Box}{name=AGB,column=0,below=AFH}{
Box for another common feature (add more boxes as needed)
}

%----------------------------------------------------------------------------------------
%	CONCLUSION
%----------------------------------------------------------------------------------------

\headerbox{Conclusion}{name=conclusion,column=0,below=AGB}{
These technologies have changed the world as we know it, allowing for us to be interconnected to a much greater extent than any time before. Many people's livelihoods, and some of the largest industries on the planet, revolve around the use of waves for communication.
}

%----------------------------------------------------------------------------------------
%	REFERENCES
%----------------------------------------------------------------------------------------

\headerbox{References}{name=references,column=0,below=conclusion}{

\smaller % Reduce the font size in this block
idk if we need this
\renewcommand{\section}[2]{\vskip 0.05em} % Get rid of the default "References" section title
\nocite{*} % Insert publications even if they are not cited in the poster

\bibliographystyle{unsrt}
\bibliography{sample} % Use sample.bib as the bibliography file
}

%----------------------------------------------------------------------------------------
%	ACKNOWLEDGEMENTS
%----------------------------------------------------------------------------------------

\headerbox{Acknowledgements}{name=acknowledgements,column=0,below=references, above=bottom}{
\smaller % Reduce the font size in this block
Rito
} 

%----------------------------------------------------------------------------------------
%	Wi-Fi
%----------------------------------------------------------------------------------------
\headerbox{Wi-Fi}{name=wifi,span=2,column=1,row=0}{ % To reduce this block to 1 column width, remove 'span=2'

\smaller
\smaller

\textbf{What is Wi-Fi}
We use Wi-Fi to access internet. Wi-Fi is a trademarked name that means IEEE 802.11x. It was created by the Wi-Fi Alliance and they define Wi-Fi as any “wireless local area network (WLAN) products that are based on the Institute of Electrical and Electronics Engineers' (IEEE) 802.11 standards.". Wi-Fi is usually a wave of frequency between 2.4GHz to 5GHz.  This frequency is considerably higher than the frequencies used for cell phones, walkie-talkies and televisions. The higher frequency allows the signal to carry more data. All waves are EM (electromagnetic).

\textbf{Types of Wi-Fi}
802.11a is a 5GHz wave capable of transmitting 54 mbps.
802.11b is a 2.4GHz wave capable of transmitting 11 mbps.
802.11g is a 2.4GHz wave capable of transmitting 54 mbps.
802.11n is a 2.4GHz wave capable of transmitting 24 - 150 mbps.
802.11ac is a 5GHz wave capable of transmitting 450 mbps.

\textbf{Uses}
Networks are created through Wi-Fi where multiple devices can connect to one source of Wi-Fi and thus access the internet as well as communicate with other devices connected to the network. 

\smaller
\textbf{Advantages}
•	The wireless nature of such networks allows users to access network resources from nearly any convenient location within their primary networking environment 
•	Users connected to a wireless network can maintain a nearly constant affiliation with their desired network as they move from place to place
•	Initial setup of an infrastructure-based wireless network requires little more than a single access point.
•	Wireless networks can serve a suddenly increased number of clients with the existing equipment
•	Wireless networking hardware is at worst a modest increase from wired counterparts

\textbf{Disadvantages}
•	To combat this consideration, wireless networks may choose to utilize some of the various encryption technologies available
•	The typical range of a common 802.11g network with standard equipment is on the order of tens of meters
•	Like any radio frequency transmission, wireless networking signals are subject to a wide variety of interference, as well as complex propagation effects that are beyond the control of the network administrator
•	The speed on most wireless networks (typically 1-54 Mbps) is far slower than even the slowest common wired networks (100Mbps up to several Gbps)
}

%----------------------------------------------------------------------------------------
%	Optical Fibres
%----------------------------------------------------------------------------------------
\headerbox{Optical Fibres}{name=optical_fibre,span=2,column=1,below=wifi}{

\textbf {Introduction}
Optical fibres utilise total internal reflection to confine light rays within its core. Modern fibre technologies are limited by physical phenomena of light travelling in an optical medium.

\textbf{Residual Absorption}
Fundamental vibration frequencies of the particles that make up the glass absorbs light with matching frequencies.   

\textbf {Dispersion}
Dispersion is an optical phenomenon where light of different frequencies travel at different velocities through an optical medium. In optical communications, data is coded in binary form and transmitted as pulses of light. As a laser pulse emits more than a single frequency of light, it is critical that the gap in time at the receiving end is not greater than the time period of the wave group, otherwise the original data would be lost. 

\textbf {Rayleigh Scattering}
An atom or molecule reradiates incident light in any direction except the incident direction. This effect is magnified at shorter wavelengths, and is increased by imperfections in the composition of the silica glass on a molecular level. 

}

%----------------------------------------------------------------------------------------
%	Bluetooth
%----------------------------------------------------------------------------------------
\headerbox{Bluetooth {\includegraphics[scale=0.022]{bluetooth_logo}}}{name=bluetooth,span=2,column=1,below=optical_fibre}{
\textbf{History}
Bluetooth was developed by the Swedish telephone company Ericsson AB in
1990

\textbf{Master/Slave Topology}
Bluetooth follows a master/slave topology where there is a master device broadcasting data to a maximum of seven slave devices. This network of 8 devices is
known as a piconet. The master will always default to being the device which
initialised the connection, however master and slave roles can be exchanged
given that both devices agree upon this. 
{\includegraphics[scale=0.12]{master_slave_topology}} % University/lab logo

\textbf{AFH}
Bluetooth uses a technique known as AFH, which is explained on the left side of this poster.
}

%----------------------------------------------------------------------------------------
%	Li-Fi
%----------------------------------------------------------------------------------------
\headerbox{Li-Fi}{name=lifi,span=2,column=1,below=bluetooth}{

%\textbf{Definition} Li-Fi is transmission of data by taking the fiber out of fiber optics by sending data through an LED bulb which varies in intensity faster than the human eye can follow.

\textbf{Principle}  A light emitter, a photo detector; Modulate light intensity faster than eyes can follow; Receiver dongle converts  changes electrical signals; Signals converted back into a data stream and transferred to a mobile device.
                                   
\begin{wrapfigure}{r}{0.22\textwidth}
	{\includegraphics[scale=0.15]{Lifi}}
\end{wrapfigure}                                             
 \textbf{Advantages} 1.faster,safer and more effiecient and diverse transmission of data 2. Bandwidth of visible spectrum is 10,000 times more than that of radio spectrum 3. Implementation and maintenance costs are minimal compared to Wi-Fi. 4.	Elimination of health problems 5. Enabling of Internet of Things on a large scale



\textbf{Challenges} 1. Light can’t pass through object  2. Interference of other light sources can cause interruption in communication. 3. Communication is limited to point-to-point transfer when implemented at very  high frequencies.



}

%----------------------------------------------------------------------------------------
%	NFC
%----------------------------------------------------------------------------------------
\headerbox{Near Field Communication}{name=nfc,span=2,column=1,below=lifi}{
\textbf{How It Works}
Near Field Communication (NFC) works off of small chips enabling data transfer between devices, of which there are active and passive ones. Active NFC devices are powered by an external source, and are able to both send and receive data. Passive devices can only send data, but do not require a power source of any kind. In close proximity with one another, the electromagnetic field of the active device will induce a small current in the passive one. 

\textbf{Stats}\\
Max Range = 20cm. Max Speed = 424kbit/s. Transmission Frequency = 13.56MHz

\textbf{Advantages}\\
Power Efficiency. The passive devices do not require a power source, only active ones do.

Control. Since the range is small, users will mostly be aware and be able to determine when exactly information transfer occurs.

Convenience. NFC does not require any prior setup or pairing to use, unlike bluetooth or wifi. Data is transferred the moment the devices are within range, making it the best method for quick small transfers.

\textbf{Disadvantages}\\
Transfer Speed. The maximum data transfer speed for NFC devices is 424kbit/s. This is significantly lower than other forms of communication. 

}

%----------------------------------------------------------------------------------------
%	Neutrinos
%----------------------------------------------------------------------------------------
\headerbox{Neutrino}{name=neutrino,span=2,column=1,below=nfc,above=bottom}{

\textbf{History}
Neutrino messaging is a hypothetical form of communication currently undergoing research. 
It was first experimentally verified to work in 2012 by researchers from the
University of Rochester and North Carolina State University.\\
\textbf{Advantages}
Unlike traditional forms of communication which rely on electromagnetic radiation, neutrinos are
affected only by the weak force and gravity, meaning they can pass messages through virtually
anything.
This can be utilised to transmit information across vast expanses in space, or for a more present-day application,
to send messages to nuclear submarines, as seawater can obstruct
electromagnetic radiation.\\
\textbf{Disadvantages}
The uninteractive nature of neutrinos causes them to be difficult to detect. Neutrinos also oscillate
between 3 flavours - electron, muon, and tau - this can be represented by a neutrino switching
between waves of different frequencies as it travels through space. This can be a
problem for certain detection methods.

}
\end{poster}

\end{document}